\documentclass{article}
\usepackage{graphicx} % Required for inserting images

\title{H5}
\author{Ben Henderson}
\date{March 2025}

\begin{document}

\maketitle

\vspace{5mm}

\section{}

The system dynamics are described by the following system of differential equations

\begin{center}
\dot{\alpha} = -0.31\alpha + 57.4r + 0.232\delta, \\
\dot{r} = -0.016\alpha - 0.425r + 0.0203\delta,  \\
\dot{\theta} = 56.7r \\
\end{center}

Now applying the Laplace transform

\begin{center}
\begin{align*}
sA(s) &= -0.31A(s) + 57.4R(s) + 0.232\Delta(s), \\
sR(s) &= -0.016A(s) - 0.425R(s) + 0.0203\Delta(s), \\
s\Theta(s) &= 56.7R(s)
\end{align*}
\end{center}


\vspace{20mm}

Now for the transfer function where the pitch angle is output and the elevator deflection angle is input.

\vspace{5mm}

\[
sA(s) + 0.31A(s) =  57.4R(s) + 0.232\Delta(s)
\]

\[
A(s) = \frac{57.4R(s) + 0.232\Delta(s)}{s + 0.31} \\
s\Theta(s) = 56.7R(s)
\]

\[
R(s) = \frac{s\Theta(s)}{56.7}
\]
\[
sR(s) = -0.016A(s) - 0.425R(s) + 0.0203\Delta(s)
\]

\[
\frac{s^2\Theta(s)}{56.7} = -0.016\frac{57.4\frac{s\Theta(s)}{56.7} + 0.232\Delta(s)}{s + 0.31} - 0.425\frac{s\Theta(s)}{56.7}+0.0203\Delta(s)
\]


\[
s^2\Theta(s) = -0.016\frac{57.4s\Theta(s) + (56.7)(0.232)\Delta(s)}{s + 0.31} - 0.425s\Theta(s)+(56.7)(0.0203)\Delta(s)
\]

\[
s^2\Theta(s) + 0.016\frac{57.4s\Theta(s)}{s + 0.31}  + 0.425s\Theta(s) = -0.016\frac{(56.7)(0.232)\Delta(s)}{s + 0.31} +(56.7)(0.0203)\Delta(s)
\]

\[
(s^2 + 0.016\frac{57.4s}{s + 0.31}  + 0.425s)(\Theta(s)) = (-0.016\frac{(56.7)(0.232)}{s + 0.31} +(56.7)(0.0203))\Delta(s)
\]

\[
\frac{\Theta(s)}{\Delta(s)} = \frac{(-0.016\frac{(56.7)(0.232)}{s + 0.31} +(56.7)(0.0203))}{(s^2 + 0.016\frac{57.4s}{s + 0.31}  + 0.425s)}
\]

\[
\frac{\Theta(s)}{\Delta(s)} = \frac{\frac{-0.210}{s + 0.31} + 1.151}{s^2 + \frac{0.918s}{s + 0.31}  + 0.425s}
\]

\[
G_s(s) = \frac{\Theta(s)}{\Delta(s)} = \frac{1.151s + 0.147}{s^3 + 0.735s^2 + 1.05s}
\]

\vspace{65mm}

Then form the transfer function for the actuator

\vspace{5mm}

\[
G_a(s) = \frac{1}{0.0145s + 1}
\]
\vspace{5mm}

Now that this has been determined, consider the transfer function for the sensor

\[G_m(s) = \frac{\exp(-0.0063s)}{0.0021s + 1}\]


\vspace{5mm}

Then define the controller transfer function, as a PID controller 
\[
G_c(s) = K_p + \frac{K_i}{s} + K_d s
\]

\vspace{5mm}


Outlining the Open-Loop Dynamics

\[
G_{ol}(s) = \frac{X_m(s)}{U(s)} = G_c(s) \cdot G_a(s) \cdot G(s) \cdot G_m(s)
\]

\vspace{5mm}



\begin{figure}[h]
    \centering
    \includegraphics[width=1\linewidth]{Screenshot from 2025-03-27 09-48-37.png}
    \caption{Block diagram of the control system for analysis with system inputs $U(s)$ and zero disturbance.}
    \label{fig:enter-label}
\end{figure}


\vspace{55mm}

\begin{figure}[h]
    \centering
    \includegraphics[width=1\linewidth]{image.png}
    \caption{Block diagram of the control system for analysis with disturbance $D(s)$ where set point is zero.}
    \label{fig:enter-label}
\end{figure}

Note that $G_L(s) = \frac{X(s)}{D(s)}$ and that for stabilizing under disturbance, the elevator disturbance angle is zero and the plane is perfectly levelz.




\end{document}
